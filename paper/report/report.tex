\documentclass[conference]{IEEEtran}
\usepackage{cite}
\usepackage{amsmath,amssymb,amsfonts}
\usepackage{algorithmic}
\usepackage{graphicx, subfig}
\usepackage{textcomp}
\usepackage{xcolor}
\usepackage{listings}
\usepackage{hyperref}
\usepackage{float}

\def\BibTeX{{\rm B\kern-.05em{\sc i\kern-.025em b}\kern-.08em T\kern-.1667em\lower.7ex\hbox{E}\kern-.125emX}}

\begin{document}

\title{Manual ADS-B Communication Protocol Implementation on the ADALM-PLUTO Using GNURadio}

\author{
    \IEEEauthorblockN{Alan Manuel Loreto Cornídez}
    \IEEEauthorblockA{\textit{College of Electrical and Computer Engineering} \\
    \textit{The University of Arizona}\\
    Tucson, Arizona \\
    aloretocornidez@arizona.edu}
\and
    \IEEEauthorblockN{Jeremy Sharp}
    \IEEEauthorblockA{\textit{College of Electrical and Computer Engineering} \\
    \textit{The University of Arizona}\\
    Tucson, Arizona \\
    jeremysharp@arizona.edu}
}

\maketitle

\begin{IEEEkeywords}
  Software Defined Radio, ADS-B, GNURadio
\end{IEEEkeywords}

\begin{abstract}
%% Write Abstract Here
\end{abstract}



\section{Introduction}
%% Right now this is the proposal.
The most common method for aircraft to report their system state involves the use of the Automatic Dependent Surveillance-Broadcast (ADS-B) transmission method. An open transmission method used to broadcast an aircraft's position, enabling the ability to track the aircraft.

Throughout the Spring 2024 semester, we have worked with the ADA Pluto Software Defined Radio (SDR) to receive and transmit signals in multiple ranges of frequency bands. We have implemented pre-made signal processing blocks in GNURadio signal flow graphs and then implemented them on the ADA Pluto for applications such as FM Radio and AM radio. Our project would like to work on solving the "1090 MHz Riddle".

What is the 1090 MHz Riddle you may ask? For many Software Defined Radio (SDR) enthusiasts, being able to capture, decode, interpret, and transmit ADS-B signals involves an understanding of how signals are are manipulated in the RF spectrum, both for transmission and modulation.

In our particular case, we would like to implement the use of the 1090 MHz frequency band to communicate with public aircraft information.

We have found pre-made GNURadio block that implement ADS-B communication protocols, however, we would like to explore a custom implementation using only a GNURadio signal flow graph and the QT GUI elements.

After implementing a simple receiver and transmitter, we would like to expand the project scope to receive real ADS-B signals transmitted by aircraft. If this project goal is met, we would like to decode and interpret the data by plotting received data on a map.

\section{Related Work}

Checking Citation and then writing.\cite{testCite}

\subsection{Convolution}
\section{Methodology}
\section{Conclusion}



%% Generate the BibTeX bibliography.
\bibliographystyle{ieeetr}
\bibliography{refs}

\end{document}

